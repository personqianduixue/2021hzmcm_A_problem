文件名:scatter\_color\_img\_rgb.m,功能:绘制图像1,2的rgb散点图,

scatter\_color\_cizhuan\_rgb.m同理
\begin{lstlisting}[language=matlab]
figure('Position',[193.8,305,908.8,366.4])
subplot(121)
data = load('img1.txt');
color = data(:,2:4);
scatter3(color(:,1),color(:,2),color(:,3),50,color./255,'filled');hold on 
grid on
ax = gca;
XYZ = 1:255;
num255 = ones(1,255)*255;
plot3(num255,num255,XYZ,'k',num255,XYZ,num255,'k',XYZ,num255,num255,'k');
ax.XAxisLocation = 'origin';
ax.YAxisLocation = 'origin';
xlim([0,256]),ylim([0,256]),zlim([0,256])
view(124,38)
xlabel('R');ylabel('G');zlabel('B');
title('图像1 RGB颜色散点图')
subplot(122)
data = load('img2.txt');
color = data(:,2:4);
scatter3(color(:,1),color(:,2),color(:,3),50,color./255,'filled');hold on 
grid on
ax = gca;
XYZ = 1:255;
num255 = ones(1,255)*255;
plot3(num255,num255,XYZ,'k',num255,XYZ,num255,'k',XYZ,num255,num255,'k');
ax.XAxisLocation = 'origin';
ax.YAxisLocation = 'origin';
xlim([0,256]),ylim([0,256]),zlim([0,256])
view(124,38)
xlabel('R');ylabel('G');zlabel('B');
title('图像2 RGB颜色散点图')
print(gcf,'..\img\图像1 2的rgb颜色','-dpng','-r600')
\end{lstlisting}	
文件名:scatter\_color\_img\_Lab.m,功能:绘制图像1,2的Lab散点图,

scatter\_color\_cizhuan\_Lab.m同理
\begin{lstlisting}[language=matlab]
figure('Position',[193.8,305,908.8,366.4])
subplot(121)
data = load('img1.txt');
color = data(:,2:4);
colorlength = size(color,1);
lab = [];
for i = 1:colorlength
	lab = [lab;rgb2lab(color(i,:)/255)];
end
scatter3(lab(:,2),lab(:,3),lab(:,1),50,color./255,'filled');hold on 
xlabel('a');ylabel('b');zlabel('L');
xlim([-128,128]),ylim([-128,128]),zlim([0,100])
title('图像1 lab颜色')
subplot(122)
data = load('img2.txt');
color = data(:,2:4);
colorlength = size(color,1);
lab = [];
for i = 1:colorlength
	lab = [lab;rgb2lab(color(i,:)/255)];
end
scatter3(lab(:,2),lab(:,3),lab(:,1),50,color./255,'filled');hold on 
xlabel('a');ylabel('b');zlabel('L');
xlim([-128,128]),ylim([-128,128]),zlim([0,100])
title('图像2 lab颜色')
print(gcf,'..\img\图像1 2的lab颜色','-dpng','-r600')
\end{lstlisting}




文件名:color\_similarity\_DE2000\_rgb.m,功能:使用DE2000算法计算相似点,并在RGB颜色空间绘制相似网络,

color\_similarity\_DE2000\_lab.m,color\_similarity\_DE76\_rgb.m,color\_similarity\_DE76\_lab.m同理
\begin{lstlisting}[language=matlab]
clc;clear
cizhuan_rgb = load('瓷砖.txt');
img1_rgb = load('img1.txt');
img2_rgb = load('img2.txt');
cizhuan_rgb = cizhuan_rgb(:,2:4);
img1_rgb = img1_rgb(:,2:4);
img2_rgb = img2_rgb(:,2:4);
cizhuan_length = size(cizhuan_rgb,1);
img1_length = size(img1_rgb,1);
img2_length = size(img2_rgb,1);
DE_img1 = ones(img1_length,cizhuan_length);
fun = @deltaE; %DE76
% fun = @(rgb1,rgb2) imcolordiff(rgb1,rgb2,"Standard","CIEDE2000");%DE2000
for i = 1:img1_length
	each_img_rgb = img1_rgb(i,:);
	for j = 1:cizhuan_length
		each_cizhuan_rgb = cizhuan_rgb(j,:);
		DE_img1(i,j) = fun(uint8(each_img_rgb),uint8(each_cizhuan_rgb));%计算相似度
	end
end
DE_img1
for i = 1:img1_length
	position_img1(i) = find(DE_img1(i,:)==min(DE_img1(i,:)));%图像中每种rgb对应瓷砖的rgb的编号
end
result1 = [1:img1_length;position_img1]';
dlmwrite('.\DE76\result1.txt',result1,'delimiter',',','newline','pc')

for i = 1:img2_length
	each_img_rgb = img2_rgb(i,:);
	for j = 1:cizhuan_length
		each_cizhuan_rgb = cizhuan_rgb(j,:);
		DE_img2(i,j) = fun(uint8(each_img_rgb),uint8(each_cizhuan_rgb));%计算相似度
	end
end
DE_img2
for i = 1:img2_length
	position_img2(i) = find(DE_img2(i,:)==min(DE_img2(i,:)));%图像中每种rgb对应瓷砖的rgb的编号
end
result2 = [1:img2_length;position_img2]';
dlmwrite('.\DE76\result2.txt',result2,'delimiter',',','newline','pc')

figure(1)
scatter3(cizhuan_rgb(:,1),cizhuan_rgb(:,2),cizhuan_rgb(:,3),100,cizhuan_rgb./255,'filled');hold on
scatter3(img1_rgb(:,1),img1_rgb(:,2),img1_rgb(:,3),25,img1_rgb./255,'filled');hold on 
for i = 1:img1_length
	img_rgb = img1_rgb(i,:);
	cizhuanrgb = cizhuan_rgb(position_img1(i),:);
	plot3([img_rgb(1),cizhuanrgb(1)],[img_rgb(2),cizhuanrgb(2)],[img_rgb(3),cizhuanrgb(3)],'color',...
	cizhuanrgb/255)
end
ax = gca;
ax.XAxisLocation = 'origin';
ax.YAxisLocation = 'origin';
xlim([0,256]),ylim([0,256]),zlim([0,256])
view(101,32.8)
xlabel('R');ylabel('G');zlabel('B');
title('图像1与瓷砖rgb颜色匹配网络(DE76)')
print(gcf,'..\img\图像1与瓷砖rgb颜色匹配网络DE76','-dpng','-r600')

figure(2)
scatter3(cizhuan_rgb(:,1),cizhuan_rgb(:,2),cizhuan_rgb(:,3),100,cizhuan_rgb./255,'filled');hold on
scatter3(img2_rgb(:,1),img2_rgb(:,2),img2_rgb(:,3),25,img2_rgb./255,'filled');hold on 
for i = 1:img2_length
	img_rgb = img2_rgb(i,:);
	cizhuanrgb = cizhuan_rgb(position_img2(i),:);
	plot3([img_rgb(1),cizhuanrgb(1)],[img_rgb(2),cizhuanrgb(2)],[img_rgb(3),cizhuanrgb(3)],'color',...
	cizhuanrgb/255)
end
ax = gca;
ax.XAxisLocation = 'origin';
ax.YAxisLocation = 'origin';
xlim([0,256]),ylim([0,256]),zlim([0,256])
view(101,32.8)
xlabel('R');ylabel('G');zlabel('B');
title('图像2与瓷砖rgb颜色匹配网络(DE76)')
print(gcf,'..\img\图像2与瓷砖rgb颜色匹配网络DE76','-dpng','-r600')
\end{lstlisting}

文件名:add\_color.m,功能:计算问题2的整数规划问题

\begin{lstlisting}[language=matlab]
clc,clear
cizhuan_rgb = load('瓷砖.txt');
cizhuan_rgb = cizhuan_rgb(:,2:4);
NUM =  size(cizhuan_rgb,1);

n = 0;
diary(['log_增加',num2str(n),'.log']);diary on;
R = intvar(1,n);G = intvar(1,n);B = intvar(1,n);
for i = 1:n
cizhuan_rgb = [cizhuan_rgb;[R(i),G(i),B(i)]];
end

for i = 1:NUM+n
	for j = 1:NUM+n
		if i>NUM||j>NUM
			D(i,j) = color_similarity(cizhuan_rgb(i,:),cizhuan_rgb(j,:));
	end
end
end
for i =1:NUM
	for j =1:NUM
		D(i,j) = color_similarity(cizhuan_rgb(i,:),cizhuan_rgb(j,:));
	end
end
D = D+diag(ones(1,n+NUM)*inf);

C = [];
for i = 1:n
	C =[C,R(i)>=0,R(i)<=255,G(i)>=0,G(i)<=255,B(i)>=0,B(i)<=255];
end
Z = -sum((min(D)));

result  = optimize(C,Z);
% if result.problem== 0
%     value(R)
%     value(G)
%     value(B)
%     value(Z)
% else
%     disp('求解过程中出错');
% end
value(R)
value(G)
value(B)
value(Z)
%% 绘图
figure('Position',[187.4,300.2,884.8,392.8])
data = load('瓷砖.txt');
color = data(:,2:4);
colorlength = size(color,1);
subplot(121)
scatter3(color(:,1),color(:,2),color(:,3),25,color./255,'filled');hold on
grid on
ax = gca;
XYZ = 1:255;
num255 = ones(1,255)*255;
plot3(num255,num255,XYZ,'k',num255,XYZ,num255,'k',XYZ,num255,num255,'k');hold on
ax.XAxisLocation = 'origin';
ax.YAxisLocation = 'origin';
xlim([0,256]),ylim([0,256]),zlim([0,256])
view(-37.5+160,28)
xlabel('R');ylabel('G');zlabel('B');title(['RGB(添加',num2str(n),'个点)'])
scatter3(value(R),value(G),value(B),100,[value(R)',value(G)',value(B)']/255,'filled')

subplot(122)
lab = rgb2lab(color/255);
scatter3(lab(:,2),lab(:,3),lab(:,1),25,color./255,'filled');hold on
grid on
xlim([-128,128]),ylim([-128,128]),zlim([0,100])
xlabel('a');ylabel('b');zlabel('L');
out_rgb = [value(R)',value(G)',value(B)'];
out_lab = rgb2lab(out_rgb/255);
title(['Lab(添加',num2str(n),'个点)'])
scatter3(out_lab(:,2),out_lab(:,3),out_lab(:,1),100,out_rgb./255,'filled');
print(gcf,['..\img\添加',num2str(n),'个点'],'-dpng','-r600')
%% 保存数据

value_RGB = num2cell([(1:n)',value(R)',value(G)',value(B)']);
title = {'序号','R','G','B'};
result_RGB = [title;value_RGB];
writecell(result_RGB,['.\增加',num2str(n),'个点.xlsx']);
dlmwrite('目标值.xlsx',-value(Z),'-append');
diary off;

\end{lstlisting}


文件名:multiobj.m,multiobj\_solve.m,功能:使用遗传算法求解多目标规划问题

\begin{lstlisting}[language=matlab]
function f =multiobj(x)
f(2) = x;
f(1) = -(1.456*x-0.1841)./(x+7.622);
\end{lstlisting}

\begin{lstlisting}[language=matlab]
clear
clc
fitnessfcn = @multiobj;   % Function handle to the fitness function
nvars = 1;                      % Number of decision variables
lb = [0];                   % Lower bound
ub = [50];                     % Upper bound
A = []; b = [];                 % No linear inequality constraints
Aeq = []; beq = [];             % No linear equality constraints
options = gaoptimset('ParetoFraction',0.3,'PopulationSize',100,'Generations',200,'StallGenLimit',200,'TolFun',1e-100,'PlotFcns',@gaplotpareto);

[x,fval] = gamultiobj(fitnessfcn,nvars, A,b,Aeq,beq,lb,ub,options);
\end{lstlisting}


文件名:shang.m,功能:使用熵权法求解权重,进而求解最优解

\begin{lstlisting}[language=matlab]
clc;clear
beishu = 1;
maxc = 100;
figure('Position',[129.8,294.6,967.2,380])
plotlegend = []; 
i = 0;
for maxcolornum = 20:10:maxc
i = i+1;
colornum = 1:maxcolornum;
biaoxianli = (1.456*colornum-0.1841)./(colornum+7.622);
chenben = beishu*colornum;
chenben = mapminmax(chenben,0,1);
biaoxianli = mapminmax(biaoxianli,0,1);
%加载数据  列数表示指标数 , 行数表示评价的个体数 
%此数据 7个评价个体 3个评价指标
X = [chenben',biaoxianli'] ;
%说明指标是正向指标还是负向指标
%此数据第一个是负向指标, 其余为正向指标
Ind=[2 1]; %Specify the positive or negative direction of each indicator
%S 为分数排名 W为指标权重
[S,W]=sqf(X,Ind); % get the score

func = @(x) (1.456*x-0.1841)./(x+7.622);
result = -W(1)*mapminmax(colornum)+W(2)*mapminmax(func(colornum));
plotnow = plot(colornum,result,'LineWidth',1.5);
plotlegend = [plotlegend,plotnow];
legendstr{i} = ['maxnum=',num2str(maxcolornum)];
hold on;
[maxvalue,position] = max(result);
scatter(position,maxvalue,'*')
line([position,position],[-0.8,maxvalue],'LineStyle','--','LineWidth',0.5)

end
% xtick([0,5,6,7,10:10:maxc])
% xticklabel([0,5,6,7,10:10:maxc])
set(gca,'XTick',[0,4,5,6,10:10:maxc]);
set(gca,'XTickLabel',[0,4,5,6,10:10:maxc]);
lgd = legend(plotlegend,legendstr,'NumColumns',2);
title(lgd,'不同的最大新增颜色数')
% legend('boxoff')
title('得分随新增颜色数量的变化');xlabel('新增颜色数量');ylabel('相对得分')
print(gcf,'..\img\得分随新增颜色数量的变化','-dpng','-r600')

function [s,w]=sqf(x,ind)
%实现用熵值法求各指标(列)的权重及各数据行的得分
%x为原始数据矩阵, 一行代表一个样本, 每列对应一个指标
%ind指示向量,指示各列正向指标还是负向指标,1表示正向指标,2表示负向指标
%s返回各行(样本)得分,w返回各列权重
[n,m]=size(x); % n个样本, m个指标
%%数据的归一化处理
for i=1:m
if ind(i)==1 %正向指标归一化
X(:,i)=guiyi(x(:,i),1,0.002,0.996);    %若归一化到[0,1], 0会出问题
else %负向指标归一化
X(:,i)=guiyi(x(:,i),2,0.002,0.996);
end
end

Y = X;
%%计算第j个指标下,第i个样本占该指标的比重p(i,j)
for i=1:n
for j=1:m
p(i,j)=X(i,j)/sum(X(:,j));
end
end

%%计算第j个指标的熵值e(j)
k=1/log(n);
for j=1:m
e(j)=-k*sum(p(:,j).*log(p(:,j)));
end
d=ones(1,m)-e; %计算信息熵冗余度
w=d./sum(d); %求权值w
s=100*w*p'; %求综合得分
end

function y=guiyi(x,type,ymin,ymax)
%实现正向或负向指标归一化,返回归一化后的数据矩阵
%x为原始数据矩阵, 一行代表一个样本, 每列对应一个指标
%type设定正向指标1,负向指标2
%ymin,ymax为归一化的区间端点
[n,m]=size(x);
y=zeros(n,m);
xmin=min(x);
xmax=max(x);
switch type
case 1
for j=1:m
y(:,j)=(ymax-ymin)*(x(:,j)-xmin(j))/(xmax(j)-xmin(j))+ymin;
end
case 2
for j=1:m
y(:,j)=(ymax-ymin)*(xmax(j)-x(:,j))/(xmax(j)-xmin(j))+ymin;
end
end
end
function X0 = stand(X)

zeros(size(X));
[nr,nx]=size(X);
for mk=1:nr
X0(mk,:)=(X(mk,:)-mean(X))./std(X);
end
end
\end{lstlisting}
文件名:table\_colorize.m,功能:根据表格的RGB值设置单元格颜色

\begin{lstlisting}[language=matlab]
try
Excel = actxGetRunningServer('Excel.Application');
catch
Excel = actxserver('Excel.Application'); % 创建一个本地Excel服务器
end

% Excel.Visible = 1;  % 让打开的Excel可见(默认不可见)

for i = 20:20
ReadExcel = Excel.Workbooks.Open([pwd,['\增加',num2str(i),'个点.xlsx']]);
Sheets = Excel.ActiveWorkBook.Sheets;
Sheets_1 = Sheets.Item(1);

RGB = Sheets_1.Range(['B2:D',num2str(i+1)]).value
%     cell2mat(RGB(1,:))
for j = 2:i+1
	SetColor(Sheets_1,['E',num2str(j)],cell2mat(RGB(j-1,:)))
	end
	ReadExcel.SaveAs([pwd, ['\增加',num2str(i),'个点(带颜色).xlsx']]);
end

Excel.Quit;     % 关闭excel

function SetColor(sheet, strRange, dColor)
%
C_r = round(dColor(1));
C_g = round(dColor(2));
C_b = round(dColor(3));
color = hex2dec([dec2hex(C_b,2) dec2hex(C_g,2) dec2hex(C_r,2)]);
sheet.Range(strRange).Interior.Color = color;
end
\end{lstlisting}
